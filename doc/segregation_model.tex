\documentclass[a4paper,12pt]{article}


\usepackage[english]{babel}
\usepackage{ucs}
\usepackage[utf8x]{inputenc}
\usepackage{graphicx}
\usepackage{amsmath}
\usepackage{amsfonts}
\usepackage{amssymb}
\usepackage{upgreek}
\usepackage[colorlinks=True,%
linkcolor=black,%
citecolor=black,%
urlcolor=black]{hyperref}

\usepackage{natbib}
\usepackage{times}

\newcommand{\UM}{\upmu\mbox{m}}
\newcommand{\UMS}{\upmu\mbox{m}.\mbox{s}^{-1}}
\setlength{\voffset}{-1in}
\setlength{\hoffset}{-0.9in}
\setlength{\textwidth}{18cm}
\setlength{\textheight}{26cm}


\newcommand{\Pp}{prophase}
\newcommand{\PMp}{prometaphase}
\newcommand{\Mp}{metaphase}
\newcommand{\Ap}{anaphase}

\usepackage{natbib}






\title{Chromosome segregation model - detailed description}


\author{Guillaume Gay, Xavier Bressaud}
\date{}


\begin{document}
\bibliographystyle{plainnat}

\maketitle{}


\section*{Introduction}

This is a more detailed version of the kinetochore segregation model
published in \cite{gayJCB12}, which should be referred to for
all the experimental, biological and non-technical aspects of this
work.

 
\section{Definitions}
\label{sec:defs}

\subsection{State vector}

The mitotic spindle is described by the speeds and position along the
$x$ axis of two spindle pole bodies, $N$ chromosomes with two
centromeres and $M_k$ attachment sites per centromere.

Positions are noted as follow:
\begin{itemize}
\item The left and right spindle pole bodies ( SPBs ), $x_s^L$ and $x_s^R$
\item The $N$ centromeres, $x_n^A, \, x_n^B, n \in \{1,\cdots, N\}$
\item The $M_k$ attachment sites of each centromere, $x_{nm}^A, \,
  x_{nm}^B, n \in \{1, \cdots, N \}, m \in \{1, \cdots, M_k\}$
\end{itemize}
The speeds are noted with a dot: $dx / dt = \dot{x}$.

As all the interactions are assumed to be parallel to the spindle
axis, only the positions along this axis are considered, in a coordinate
system with its origin at the center of the spindle, which means that
$x_s^L(t) = - x_s^R(t)\, \forall t$.

The following interactions are considered:

\subsubsection{Forces at the right SPB :}
\begin{itemize}
\item Friction forces (viscous drag):  $F_s^f = -\mu_s \dot{x_s}^R$
\item Midzone force generators (applied at the right SPB): 
  $$F_{mid} = F_z\left(1 - (\dot{x}^R_s - \dot{x}_s^L)/V_z\right) =
  F_z\left(1 - 2\dot{x}^R_s / V_z\right) $$
\item Total kinetochore microtubules force generators:
  \begin{align}
    F_{kMT}^T = \sum_{n = 1}^{N}\sum_{m = 1}^{M_k} & - \rho_{nm}^A\,F_k\left( 1 -
      (\dot{x}^A_{nm} - \dot{x}^R_s)/V_k\right)\\
    & + \lambda_{nm}^A\,F_k\left(1 -
      (\dot{x}^A_{nm} + \dot{x}^R_s)/V_k\right)\\
    & - \rho_{nm}^B\,F_k\left( 1 -
      (\dot{x}^B_{nm} - \dot{x}^R_s)/V_k\right)\\
    & + \lambda_{nm}^A\,F_k\left(1 -
      (\dot{x}^B_{nm} + \dot{x}^R_s)/V_k\right)
  \end{align}
\end{itemize}

\subsubsection{Forces at centromere $An$}

\begin{itemize}
\item Drag: $F_c^f = -\mu_c \dot{x_n}^A$
\item Cohesin bond (Hook spring) restoring force exerted by centromere
  B: $$F_{BA} = -\kappa_c (x_n^A - x_n^B - d_0),
  \mbox{with } F_{BA} = - F_{AB}$$
\item Total visco-elastic bond between the centromere A and the attachment
  sites:
  $$ F_v^T = \sum_{m = 1}^{M_k} -\kappa_k(x_n^A - x_{nm}^A) 
  - \mu_k(\dot{x}_n^A - \dot{x}_{nm}^A) $$
\end{itemize}

\subsubsection{Forces at attachment site $Anm$}

\begin{itemize}
\item Visco-elastic bond between the centromere A and the
  attachment sites:
  $$F_v =  \kappa_k(x_n^A - x_{nm}^A) 
  + \mu_k(\dot{x}_n^A - \dot{x}_{nm}^A) $$
\item Kinetochore microtubules force generators:
  $$F_{kMT}^A = \rho_{nm}^A\,F_k\left(1 - (\dot{x}^A_{nm} -
    \dot{x}^R_s)/V_k\right) - \lambda_{nm}^A\,F_k\left(1 -
    (\dot{x}^A_{nm} - \dot{x}^L_s)/V_k\right) $$
\end{itemize}

Here, $\rho_{nm}^A$ and $\lambda_{nm}^A$ are two random variables that govern
the attachment state of the site $x_{nm}^A$, such that:
\begin{align}
  \label{eq:rholambda}
  \rho_{nm}^A &= 
  \begin{cases}
    1 &\text{if the site is attached to the right SPB}\\
    0 &\text{otherwise}\\
  \end{cases}\\
  \lambda_{nm}^A &=
  \begin{cases}
    1 &\text{if the site is attached to the left SPB}\\
    0 &\text{otherwise}\\
  \end{cases}
\end{align}

Note that $\rho_{nm}^A$ and $\lambda_{nm}^A$ are not independent, as
an attachment site can't be attached to both poles. To take this into
account, we can define the variable $p_{nm}^A = \rho_{nm}^A -
\lambda_{nm}^A$ such that:
\begin{equation}
  \label{eq:rholambda}
  p_{nm}^A = 
  \begin{cases}
    - 1 &\text{if the site is attached to the left SPB}\\
    0 &\text{if the site is not attached}\\
    1 &\text{if the site is attached to the right SPB}\\
  \end{cases}\\
\end{equation}
We have:
\begin{align}
  \lambda_{nm}^A &= p_{nm}^A\left(p_{nm}^A - 1\right)/2\\
  \rho_{nm}^A &= p_{nm}^A\left(p_{nm}^A + 1\right)/2
\end{align}
We also define $N_n^{AL}$ and $N_n^{AR}$ as the number of ktMTs of
centromere A attached to the left and right SPBs, respectively:
\begin{equation}
  \label{eq:NAL}
  N_n^{AL} = \sum_{m = 1}^{M_k}\lambda_{nm}^A \mbox{ and }%
  N_n^{AR} = \sum_{m = 1}^{M_k}\rho_{nm}^A 
\end{equation}
Note that $N_n^{AL} + N_n^{AR} \leq M_k\, \forall\, p_{nm} $
The same definitions apply for the centromere B and left SPB.

\subsection{Set of first order coupled equations}

In the viscous nucleoplasm, inertia is negligible. Newton first
principle thus reduces to: $ \sum F = 0 $, all the equations are
gathered together in the system of equations:
$$
\mathbf{A}\dot{X} + \mathbf{B}X + C = 0
$$
The vector $X$ has $1 + 2N(M_k + 1)$ elements and is defined as follow:
\begin{equation*}
  X = \{x_s^R, \{x_n^A, \{x_{nm}^A\},  x_n^B,% 
  \{x_{nm}^B \}\}\}\mbox{ with } n \in 1 \cdots N %
  \mbox{ and } m \in 1 \cdots M_k
\end{equation*}
With this order, the index of the $n^{th}$ centromere A, noted
$i_{n}^A$ is given by $i_{n}^A = 2n(M_k+1) + 2$. Similarly, we have:
$$ \begin{aligned}
i_{n}^B &= 2n(M_k+1) + M_k + 3 \\
i_{nm}^A &= 2n(M_k+1) + 3 + m \\
i_{nm}^B &= 2n(M_k+1) + M_k + 4 + m \\
\end{aligned}$$

To simplify the equations, we set $F_k$ as unit force and $V_k$ as unit
speed, thus $F_k/V_k = 1$. From the above we have:
\begin{equation}
  \begin{aligned}
    A = &% 
    \begin{pmatrix}
      a_{1,1} & \hdotsfor{1} & a_{1, i_{nm}^A} &%
      \hdotsfor{1} & a_{1, i_{nm}^B}\\
      %%%%% Centromere A %%%%
      \hdotsfor{1} & a_{i_{n}^A, i_{n}^A}& %
      a_{i_{n}^A, i_{nm}^A} & \hdotsfor{1}\\
      %%%%% Att. Site A %%%%
      a_{i_{nm}^A, 1} & a_{i_{nm}^A, i_{n}^A} & a_{i_{nm}^A, i_{nm}^A}&%
      \hdotsfor{2}\\
      %%%%% Centromere B %%%%
      \hdotsfor{3} & a_{i_{n}^B, i_{n}^B}& a_{i_{n}^B, i_{nm}^B}\\
      %%%%% Att. Site B %%%%
      a_{i_{nm}^B,1} & \hdotsfor{2} &  a_{i_{n}^B, i_{nm}^B} &%
      a_{i_{nm}^B, i_{nm}^B}\\
    \end{pmatrix}\\
    =  & \begin{pmatrix}
      %%%%% SPB %%%%
      -\mu_s - F_z/V_z - \sum (p_{nm}^A + p_{nm}^B)& \hdotsfor{1} & p_{nm}^A &%
      \hdotsfor{1} &  p_{nm}^B\\
      %%%%% Centromere A %%%%
      \hdotsfor{1} &  -\mu_c - M_k \mu_k& \mu_k & \hdotsfor{2}\\
      %%%%% Att. Site A %%%%
      p_{nm}^A & \mu_k & - \mu_k + p_{nm}^A & \hdotsfor{2}\\
      %%%%% Centromere B %%%%
      \hdotsfor{3} & -\mu_c - M_k \mu_k & \mu_k\\
      %%%%% Att. Site B %%%%
      p_{nm}^B & \hdotsfor{2} & \mu_k & - \mu_k + p_{nm}^B \\
    \end{pmatrix}, \\
    B = &%
    \begin{pmatrix}
      \,0\, & \hdotsfor{4}\\
      \hdotsfor{1} & - \kappa_c - M_k \kappa_k & \kappa_k &%
      \kappa_c & \hdotsfor{1} \\
      \hdotsfor{1} & \kappa_k & -\kappa_k &  \hdotsfor{2}\\
      \hdotsfor{1} & \kappa_c & \hdotsfor{1} &%
      -\kappa_c - M_k \kappa_k & \kappa_k \\
      \hdotsfor{3}  & \kappa_k & - \kappa_k\\
    \end{pmatrix}% 
    \quad \mbox{and} \quad C = %
    \begin{pmatrix}
      0\\
      0\\
      d_0\\
      0\\
      d_0\\
      0\\
    \end{pmatrix}
  \end{aligned}
\end{equation}

\section{Continuous time Markov chain description\\
 of the attachment -- detachment process}
\label{sec:markov}


\subsection{Attachment and detachment rates}

The attachment sites attach or detach stochastically with rates $k_a^{R/L}$
and $k_d$, i.e:
\begin{equation}
  \begin{aligned}
    p_{nm}^A = 1 &\xrightarrow{\quad k_d \quad}& p_{nm}^A = 0%
    &\xrightarrow{\quad k_a^R \quad}& p_{nm}^A = 1\\
    p_{nm}^A= -1 &\xrightarrow{\quad k_d \quad}& p_{nm}^A = 0%
    &\xrightarrow{\quad k_a^L \quad}& p_{nm}^A = -1\\
  \end{aligned}
\end{equation}
The detachment rate depends on the position of the attached site with
respect to the chromosome center:
\begin{equation}
  \label{eq:k_d}
  k_d = k_ad_\alpha / d, \mbox{with } d = \left| x^A_{nm} - %
  \left(x^A_{n}+ x^B_{n}\right) / 2 \right|
\end{equation}
The attachment rate depends on the state of the other attachment
sites:
\begin{equation}
  \label{eq:k_a}
  k_a^R = k_a\left( 1/2 + \beta\frac{N_n^{AR} - N_n^{AL}}%
    {2\left(N_n^{AR} + N_n^{AL}\right)}\right)
\end{equation}
  
\subsection{Discrete state-space approximation of the stochastic
  process}
\label{sec:Defintions}

\subsubsection{Definitions}

The coupling of $k_d$ with the mechanical aspects of the global model
prevents a straightforward description of the model as a continuous
time Markov chain. This section presents a discrete approximation, as
a first attempt.

We now consider only one chromosome with two centromeres and $M_k$
attachment sites. The state of this model is then completely specified
by the four random variables $N^{AL}, N^{AR}, N^{BL}, N^{BR}$ as
defined in equation \ref{eq:NAL}.  

We define the variables $P^A$ (and similarly $P^B$):
\begin{equation}
  \label{eq:PA}
  P^A = N^{AR} - N^{AL} = \sum_{M_k} p_m^A
\end{equation}
With $p_m^A$ as defined in equation \ref{eq:rholambda}. $P^A$ can be
viewed as the force balance at the centromere A. Each of those
variables can change of one unit at a time, reflecting the attachment
or detachment of one kinetochore microtubule.

\subsubsection{Correct and erroneous attachment}

In the above definitions, the role of centromere A and B are
completely symmetrical with respect to the left and right pole. Yet,
the whole point is to segregate centromere A to one pole and
centromere B to the other. We define the correct and erroneous
attachments $N^{AC}$ and $N^{AE}$ to take this into account.

For example, in the case of A being linked mainly toward the left SPB
and B toward the right SPB, correct attachments are given by $ N^{AC}
= N^{AL}$ and $N^{BC} = N^{BR}$. More generaly, the correct directions
are the one exerting the maximal traction force:
\begin{equation}
  \label{eq:corect_erron}
  \begin{aligned}
    \mathrm{if}\, N^{AL} +  N^{BR} \geq N^{AR} +  N^{BL}:& N^{AC}&= N^{AL},\\
                                                      &  N^{BC}&= N^{BR},\\
                                                      &  N^{AE}&= N^{AR},\\
                                                      &  N^{BE}&= N^{BL}\\
    \mathrm{if}\, N^{AL} +  N^{BR} < N^{AR} +  N^{BL}  :&  N^{AC}&= N^{AR},\\
                                                      &  N^{BC}&= N^{BL},\\
                                                      &  N^{AE}&= N^{AL},\\
                                                      &  N^{BE}&= N^{BR}\\
  \end{aligned}
\end{equation}
For commodity, we define the factor $\sigma$ such that:
\begin{equation}
  \label{eq:factS}
  \sigma =%
  \begin{cases}
    1 &\mathrm{if}\, N^{AL} +  N^{BR} \geq N^{AR} +  N^{BL}\\
    -1 &\mathrm{if}\, N^{AL} +  N^{BR} < N^{AR} +  N^{BL}\\
  \end{cases}
\end{equation}
And the state vector $Phi = (N^{AC}, N^{AE}, N^{BE}, N^{BC}$.


\subsubsection{Detachment process}

Considering for example the detachment for $N^{AL}$:
\begin{equation}
    N^{AL} \xrightarrow{\quad N^{AL}k_d \quad} N^{AL}- 1%
\end{equation}

In the mechanical model, the detachment rate $k_d$ depends on the
distance between the centromeres. Here, we consider that at
equilibrium, this distance $d_{eq}$ varies linearly with the net force
applied to the chromosome by the attached microtubules. If centromere
B goes to the right and centromere A goes to the left ($S = 1$):
$$ d_{eq} = d_0 + F_k (P^B - P^A)/\kappa_c$$
\emph{In vivo}, the distance between the centromeres
never exceeds several times the rest length $d_0$. A typical value for
$d_{max}$ (around $1.2 \UM$) is four times $d_0$ ($0.3 \UM$). Taking
$d_0$ as unit length and $F_k$ as unit force, and substituting $P^A +
P^B$ by $2M_k$ in the above equation, we have:
$$ 
d_{max} = 1 + 2M_k/\kappa_c  \simeq 4 \Rightarrow \kappa_c \simeq 3/2*M_k
$$

And thus:
\begin{equation}
  \label{eq:deq}
  d_{eq} = 1 + \sigma 3 \left(P^B - P^A\right)/2M_k
\end{equation}

We thus calculate the detachment rate $k_d$ with:
\begin{equation}
  \label{eq:def_kf_markov}
  k_d  =   k_a d_{\alpha} / (1 + \sigma \left(P^A - P^B\right)/M_k) 
\end{equation}


\subsubsection{Attachment process}


The total number of attached ktMTs at centromere A can augment of one
unit, with a rate proportional to the number of unattached sites, $M_k
 -  N^{AL} -  N^{AR} $:
\begin{equation}
  N^{AL} + N^{AR} \xrightarrow{\left(M_k -
      N^{AL}-N^{AR}\right)k_a} N^{AL}+ N^{AR} + 1%
\end{equation}
To decide whether $N^{AL}$ or $N^{AR}$ augment, we state that the
probability that it is $N^{AL}$ is given by 
$$P_L^A = 1/2 + \beta\left(N^{AL} - N^{AR}\right)/2(N^{AL} + N^{AR})$$
if $N^{AL} + N^{AR} \neq 0$ and $P_L^A = 1/2$ if $N^{AL} + N^{AR} =
0$.  Consequently:
\begin{equation}
  \begin{aligned}
    N^{AL}&\xrightarrow{P_L^A\left(M_k -
        N^{AL}-N^{AR}\right)k_a} N^{AL} + 1\\
    N^{AR}&\xrightarrow{\left(1 - P_L^A\right)\left(M_k -
        N^{AL}-N^{AR}\right)k_a} N^{AR} + 1\\
  \end{aligned}
\end{equation}


\subsubsection{Markov chain generator}
\label{sec:mcg}

The matrix $Q$ is the generator of the Markov chain. Let $\Psi_i = (
N^{AL}, N^{AR}, N^{BL}, N^{BR})$ and $\Psi_j = (
{N^{AL}}', {N^{AR}}', {N^{BL}}', {N^{BR}}')$, then the transition rate between
$\Psi_i$ and $\Psi_j$ is given by $q_{ij}$ if $i \neq j$, the diagonal
terms $q_{ii}$ are given by $q_{ii} = - \sum_{j \neq i} q_{ij} $. 

We used the above rules for the transition rates to write the matrix
$Q$ for all the acceptable vectors $\Psi$. The invariant measure of
the process is then given by the normalized  vector
$\mathbf{\mu}$ such that: $\mathbf{\mu}^TQ = 0$.


\section{Classification of the defects and analysis}

We base the analysis of the invariant measure on the classification
between correct and erroneous attachment described above (equation
\ref{eq:corect_erron}).

We follow the biological classification of the mitotic defects. With
$Phi = (N^{AC}, N^{AE}, N^{BE}, N^{BC}$, the chromosome can be either
(modulo a permutation between A and B):
\begin{itemize}
\item Unattached if $\Phi = (0,0,0,0)$,
\item Monotelic if $\Phi \in \{(0,0,0, N^{BC}),N^{BC}> 0\}$,
\item Syntelic if $\Phi \in \{(N^{AC},0,N^{BE},0),N^{AC}> 0, N^{AE}> 0\}$ or
\item Merotelic if $\Phi \in \{(N^{AC},N^{AE},N^{BC},N^{BE}), N^{AC}>
    0, N^{AE}> 0, \forall N^{BC}, N^{BE}\}  $
\end{itemize}



\subsection{Dependance on $d_{\alpha}$ and $\beta$}

\begin{figure}[htbp]
  \centering
  \includegraphics[width=4in]{total_defects}
  \caption{Fraction totale de défauts pour différentes valeurs de $d_{\alpha}$ et $\beta$}
  \label{fig:total}
\end{figure}

\begin{figure}[htbp]
  \centering
  \includegraphics[width=4in]{details_defects}
  \caption{Répartition pour différentes valeurs de $d_{\alpha}$ et $\beta$}
  \label{fig:total}
\end{figure}

\end{document}

<!-- Local IspellDict: english -->

%%% Local Variables: 
%%% mode: latex
%%% TeX-master: t
%%% End: 
