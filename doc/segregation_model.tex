\documentclass[a4paper,12pt]{article}


\usepackage[english]{babel}
\usepackage{ucs}
\usepackage[utf8x]{inputenc}
\usepackage{graphicx}
\usepackage{amsmath}
\usepackage{amsfonts}
\usepackage{amssymb}
\usepackage{upgreek}
\usepackage[colorlinks=True,%
linkcolor=black,%
citecolor=black,%
urlcolor=black]{hyperref}

\usepackage{natbib}
\usepackage{times}

\newcommand{\UM}{\upmu\mbox{m}}
\newcommand{\UMS}{\upmu\mbox{m}.\mbox{s}^{-1}}
\renewcommand{\leq}{\leqslant}
\renewcommand{\geq}{\geqslant}

\setlength{\voffset}{-1in}
\setlength{\hoffset}{-0.9in}
\setlength{\textwidth}{18cm}
\setlength{\textheight}{26cm}


\newcommand{\Pp}{prophase}
\newcommand{\PMp}{prometaphase}
\newcommand{\Mp}{metaphase}
\newcommand{\Ap}{anaphase}

\usepackage{natbib}


\title{Chromosome segregation model - detailed description}

%\date{}


\begin{document}
\bibliographystyle{plainnat}

\maketitle{}


\section*{Introduction}

This is a more detailed version of the kinetochore segregation model
to be published in the JCB article, which should be referred to for
all the experimental, biological and non-technical aspects of this
work.

 
\section{Definitions}
\label{sec:defs}

\subsection{State vector}

The mitotic spindle is described by the speeds and position along the
$x$ axis of two spindle pole bodies, $N$ chromosomes with two
centromeres and $M_k$ attachment sites per centromere.

Positions are noted as follow:
\begin{itemize}
\item The left and right spindle pole bodies ( SPBs ), $x_s^L$ and $x_s^R$
\item The $N$ centromeres, $x_n^A, \, x_n^B, n \in \{1,\cdots, N\}$
\item The $M_k$ attachment sites of each centromere, $x_{nm}^A, \,
  x_{nm}^B, n \in \{1, \cdots, N \}, m \in \{1, \cdots, M_k\}$
\end{itemize}
The speeds are noted with a dot: $dx / dt = \dot{x}$.

As all the interactions are assumed to be parallel to the spindle
axis, only the positions along this axis are considered, in a coordinate
system with its origin at the center of the spindle, which means that
$x_s^L(t) = - x_s^R(t)\, \forall t$.

\subsection{Random variables for the attachment}

We define $\rho_{nm}^A$ and $\lambda_{nm}^A$, two random variables that govern
the attachment state of the site $x_{nm}^A$, such that:
\begin{align}
  \label{eq:rholambda}
  \rho_{nm}^A &= 
  \begin{cases}
    1 &\text{if the site is attached to the right SPB}\\
    0 &\text{otherwise}\\
  \end{cases}\\
  \lambda_{nm}^A &=
  \begin{cases}
    1 &\text{if the site is attached to the left SPB}\\
    0 &\text{otherwise}\\
  \end{cases}
\end{align}

Note that $\rho_{nm}^A$ and $\lambda_{nm}^A$ are not independent, as
an attachment site can't be attached to both poles. To take this into
account, we can define the variable $\pi_{nm}^A = \rho_{nm}^A -
\lambda_{nm}^A$ such that:
\begin{equation}
  \label{eq:pnm}
  \pi_{nm}^A = 
  \begin{cases}
    - 1 &\text{if the site is attached to the left SPB}\\
    0 &\text{if the site is not attached}\\
    1 &\text{if the site is attached to the right SPB}\\
  \end{cases}\\
\end{equation}
We have:
\begin{align}
  \lambda_{nm}^A &= \pi_{nm}^A\left(\pi_{nm}^A - 1\right)/2\\
  \rho_{nm}^A &= \pi_{nm}^A\left(\pi_{nm}^A + 1\right)/2
\end{align}
We also define $N_n^{AL}$ and $N_n^{AR}$ as the number of ktMTs of
centromere A attached to the left and right SPBs, respectively:
\begin{equation}
  \label{eq:NAL}
  N_n^{AL} = \sum_{m = 1}^{M_k}\lambda_{nm}^A \mbox{ and }%
  N_n^{AR} = \sum_{m = 1}^{M_k}\rho_{nm}^A 
\end{equation}
Note that $N_n^{AL} + N_n^{AR} \leq M_k\, \forall\, \pi_{nm} $
The same definitions apply for the centromere B and left SPB.

\subsection{Forces}

The following force balances are considered:

\subsubsection{Forces at the right SPB :}
\begin{itemize}
\item Friction forces (viscous drag):  $F_f^R = -\mu_s \dot{x_s}^R$
\item Midzone force generators: 
  $$F_{mid} = F_z\left(1 - (\dot{x}^R_s - \dot{x}_s^L)/V_z\right) =
  F_z\left(1 - 2\dot{x}^R_s / V_z\right) $$
\item Total kinetochore microtubules force generators:
 $$ \begin{aligned}
    F_{kMT}^T = \sum_{n = 1}^{N}\sum_{m = 1}^{M_k} & - \rho_{nm}^A\,F_k\left( 1 -
      (\dot{x}^A_{nm} - \dot{x}^R_s)/V_k\right)\\
    & - \rho_{nm}^B\,F_k\left( 1 -
      (\dot{x}^B_{nm} - \dot{x}^R_s)/V_k\right)\\
 \end{aligned} $$
\end{itemize}

\subsubsection{Forces at the left SPB :}

Because of the reference frame definition, $\dot{x_s}^R =
-\dot{x_s}^L\,\forall t$. Here we substituted $x_s^L$ with $-x_s^R$
\begin{itemize}
\item Friction forces (viscous drag):  $F_f^L = \mu_s \dot{x_s}^R$
\item Midzone force generators: 
  $$F_{mid}^L = - F_z\left(1 - 2\dot{x}^R_s / V_z\right) $$
\item Total kinetochore microtubules force generators:
$$  \begin{aligned}
    F_{kMT}^T = \sum_{n = 1}^{N}\sum_{m = 1}^{M_k} & - \lambda_{nm}^A\,F_k\left(1 +
      (\dot{x}^A_{nm} + \dot{x}^R_s)/V_k\right)\\
    & - \lambda_{nm}^B\,F_k\left(1 +
      (\dot{x}^B_{nm} + \dot{x}^R_s)/V_k\right)
  \end{aligned}
$$
\end{itemize}


\subsubsection{Forces at centromere $An$}

\begin{itemize}
\item Drag: $F_c^f = -\mu_c \dot{x_n}^A$
\item Cohesin bond (Hook spring) restoring force exerted by
  centromere\footnote{We want the centromeres to be able to cross each
    over. In one dimension, this introduces a discontinuity. In the
    previous version, the 'swap' mechanism was solving this directly
    (as $x_A$ and $x_B$ are exchanged). This is not possible any more,
    as the 'swap' mechanism is now irrelevant, as there is no prefered
    side for a given centromere. With this model, the discontinuity is
    only to the first order. I am convinced there can be a better
    regularisation.}:
  \begin{equation}
    F_{BA} =
    \begin{cases}
      \kappa_c (x_n^B - x_n^A - d_0) &\mathrm{if}\quad d_0 < x_n^A - x_n^B\\
      0  &\mathrm{if}\quad - d_0 < x_n^A - x_n^B < d_0\\
      \kappa_c (x_n^B - x_n^A + d_0) &\mathrm{if}\quad  x_n^A - x_n^B\\
    \end{cases}
  \end{equation}
  With $F_{AB} = - F_{BA}$. 
\item Total visco-elastic bond between the centromere A and the attachment
  sites:
  $$ F_v^T = \sum_{m = 1}^{M_k} -\kappa_k(x_n^A - x_{nm}^A) 
  - \mu_k(\dot{x}_n^A - \dot{x}_{nm}^A) $$
\end{itemize}


\subsubsection{Forces at attachment site $Anm$}

\begin{itemize}
\item Visco-elastic bond between the centromere A and the
  attachment sites:
  $$F_v =  \kappa_k(x_n^A - x_{nm}^A) 
  + \mu_k(\dot{x}_n^A - \dot{x}_{nm}^A) $$
\item Kinetochore microtubules force generators:
  \begin{equation}
    \begin{aligned}
      F_{kMT}^A &= F_{kMT}^{RA} + F_{kMT}^{LA}\\
      F_{kMT}^{RA} &= \rho_{nm}^A\,F_k\left(1 - \frac{\dot{x}^A_{nm} -
          \dot{x}^R_s}{V_k}\right)\\
      F_{kMT}^{LA} &=  \lambda_{nm}^A\,F_k\left(-1 - \frac{\dot{x}^A_{nm} -
          \dot{x}^L_s}{V_k}\right)\\
    \end{aligned}
  \end{equation}
With $F_k = 1$ and $ V_k = 1$ (for now on, we are taking $F_k$ as
unit force and $V_k$ as unit speed), this gives:
\begin{equation}
 F_{kMT}^A = \rho_{nm}^A\,\left(\dot{x}^R_s - \dot{x}^A_{nm} + 1\right)%
 - \lambda_{nm}^A\,\left(\dot{x}^R_s + \dot{x}^A_{nm} + 1\right)
\end{equation}
Eventually, substituting $\lambda^A_{nm} - \rho^A_{nm}$ with $\pi_{nm}^A$ and $\lambda^A_{nm} + \rho^A_{nm}$ with $|\pi_{nm}^A|$:
\begin{equation}
    F_{kMT}^A =  \pi_{nm}^A(\dot{x}^R_s + 1) - |\pi_{nm}^A|\dot{x}^A_{nm}
\end{equation}
\end{itemize}

\subsection{Set of first order coupled equations}

In the viscous nucleoplasm, inertia is negligible. Newton first
principle thus reduces to: $ \sum F = 0 $. This force balance equation
can be written for each elements of the spindle. 
To simplify further, the equations for the right and left SPBs can be
combined:
 
\begin{equation}
  \begin{aligned}
    - \mu_s\dot{x}^R_s + F_{z}\left(1 - 2\dot{x}^R_s/V_z\right)%
    + \sum_{n,m} - \rho_{nm}^A\,\left(\dot{x}^R_s - \dot{x}^A_{nm} +%
      1\right) &= 0 \, \mbox{for the right SPB}\\
    \mu_s\dot{x}^R_s - F_{z}\left(1 - 2\dot{x}^R_s/V_z\right)%
    + \sum_{n,m} \lambda_{nm}^A\,\left(\dot{x}^R_s + \dot{x}^A_{nm} +%
      1\right) &= 0 \, \mbox{for the left SPB}\\
  \end{aligned}
\end{equation}
The difference of those two expressions gives, with the same substitutions as before:

\begin{equation}
  \label{eq:spindle_term}
  - 2\mu_s\dot{x}^R_s + 2F_{z}\left(1 - 2\dot{x}^R_s/V_z\right)%
  + \sum_{n,m}- (|\pi_{nm}^A|  + |\pi_{nm}^B|)(\dot{x}^R_s + 1)%
  + \pi_{nm}^A \dot{x}_{nm}^A + \pi_{nm}^B = 0%
\end{equation}

All the equations are gathered together in the system of equations:
$$
\mathbf{A}\dot{X} + \mathbf{B}X + C = 0
$$

The vector $X$ has $1 + 2N(M_k + 1)$ elements and is defined as
follow\footnote{Note that the left SPB is omitted in $X$.}:
\begin{equation*}
  X = \{x_s^R, \{x_n^A, \{x_{nm}^A\},  x_n^B,% 
  \{x_{nm}^B \}\}\}\mbox{ with } n \in 1 \cdots N %
  \mbox{ and } m \in 1 \cdots M_k
\end{equation*}
In matrix form, we have:\\
\begin{equation}
  \begin{aligned}
    X = &%
    \begin{pmatrix}
      x_s^R\\
      x_n^A\\
      x_{nm}^A\\
      x_n^B\\
      x_{nm}^B\\
    \end{pmatrix} =%
    \begin{pmatrix}
      \text{Right SPB}\\
      \text{Centromere }A, n\\
      \text{Attachment site }A, n,m\\
      \text{Centromere }B, n\\
      \text{Attachment site }B, n,m\\
    \end{pmatrix}\\
    A = &% 
    \begin{pmatrix}
      %%%%% SPB %%%%
      - 2 \mu_s - 4 F_z/V_z - \sum (|\pi_{nm}^A| + |\pi_{nm}^B|)& \hdots & \pi_{nm}^A &%
      \hdots &  \pi_{nm}^B\\
      %%%%% Centromere A %%%%
      \hdots &  -\mu_c - M_k \mu_k& \mu_k & \hdotsfor{2}\\
      %%%%% Att. Site A %%%%
      \pi_{nm}^A & \mu_k & - \mu_k - |\pi_{nm}^A| & \hdotsfor{2}\\
      %%%%% Centromere B %%%%
      \hdotsfor{3} & -\mu_c - M_k \mu_k & \mu_k\\
      %%%%% Att. Site B %%%%
      \pi_{nm}^B & \hdotsfor{2} & \mu_k & - \mu_k - |\pi_{nm}^B| \\
    \end{pmatrix}, \\
     % = &%
    B = &%
    \begin{pmatrix}
      \,0\, & \hdotsfor{4}\\
      \hdots & - |\delta_n^{AB}|\kappa_c - M_k \kappa_k & \kappa_k &%
      |\delta_n^{AB}|\kappa_c & \hdots \\
      \hdots & \kappa_k & -\kappa_k &  \hdotsfor{2}\\
      \hdots & |\delta_n^{AB}|\kappa_c & \hdots &%
      -|\delta_n^{AB}|\kappa_c - M_k \kappa_k & \kappa_k \\
      \hdotsfor{3}  & \kappa_k & - \kappa_k\\
    \end{pmatrix}\\
    C = &%
    \begin{pmatrix}
      2Fz - \sum_{n,m}(|\pi_{nm}^A| + |\pi_{nm}^B|) \\
      \delta_n^{AB} \kappa_c d_0\\
      \pi_{nm}^A\\
      -\delta_n^{AB} \kappa_c d_0\\
      \pi_{nm}^B\\
    \end{pmatrix}
    \mathrm{with}\, \delta_n^{AB} =%      
    \begin{cases}
      1  &\mathrm{if}\quad d_0 < x_n^A - x_n^B\\
      0  &\mathrm{if}\quad - d_0 < x_n^A - x_n^B < d_0\\
      -1 &\mathrm{if}\quad  x_n^A - x_n^B\\
    \end{cases}
\end{aligned}
\end{equation}

As is actually done in the python implementation, 
$A$  can be decomposed into a time invariant part $A_0$ and a
variable part $A_t$ with:\\
\begin{equation}
  \begin{aligned}
    A_0 &=%
    \begin{pmatrix}
      - 2 \mu_s - 4 F_z/V_z & \hdotsfor{4}\\
      \hdots &  -\mu_c - M_k \mu_k& \mu_k & \hdotsfor{2}\\
      \hdots & \mu_k & - \mu_k & \hdotsfor{2}\\
      \hdotsfor{3} & -\mu_c - M_k \mu_k & \mu_k\\
      \hdotsfor{3} & \mu_k & - \mu_k\\
    \end{pmatrix}\\
    A_t &=%
    \begin{pmatrix}
      %%%%% SPB %%%%
      - \sum (|\pi_{nm}^A| + |\pi_{nm}^B|)& \hdots & \pi_{nm}^A &%
      \hdots &  \pi_{nm}^B\\
      %%%%% Centromere A %%%%
      \hdotsfor{5}\\
      %%%%% Att. Site A %%%%
      \pi_{nm}^A & \hdots & - |\pi_{nm}^A| & \hdotsfor{2}\\
      %%%%% Centromere B %%%%
      \hdotsfor{5}\\
      %%%%% Att. Site B %%%%
      \pi_{nm}^B & \hdotsfor{3} & - |\pi_{nm}^B| \\
    \end{pmatrix}\\
  \end{aligned}
\end{equation}

Equivalently for $B$:
\begin{equation}
  B_0 =% 
  \begin{pmatrix}
    \,0\, & \hdotsfor{4}\\
    \hdots &  - M_k \kappa_k & \kappa_k & \hdotsfor{2} \\
    \hdots & \kappa_k & -\kappa_k &  \hdotsfor{2}\\
    \hdots &  \hdotsfor{2} & - M_k \kappa_k & \kappa_k \\
    \hdotsfor{3}  & \kappa_k & - \kappa_k\\
  \end{pmatrix}
  =\kappa_k%
  \begin{pmatrix}
    \,0\, & \hdotsfor{4}\\
    \hdots &  - M_k  & 1 & \hdotsfor{2} \\
    \hdots & 1 & -1 &  \hdotsfor{2}\\
    \hdots &  \hdotsfor{2} & - M_k  & 1 \\
    \hdotsfor{3}  & 1 & - 1\\
  \end{pmatrix}
\end{equation}
And
\begin{equation}
  B_t =\kappa_c%
  \begin{pmatrix}
    \,0\, & \hdotsfor{4}\\
    \hdots & - |\delta_n^{AB}| & \hdots & |\delta_n^{AB}| & \hdots \\
    \hdotsfor{5}\\
    \hdots & |\delta_n^{AB}| & \hdots & -|\delta_n^{AB}|& \hdots \\
    \hdotsfor{5}\\
  \end{pmatrix}
\end{equation}


% With this order, the index of the $n^{th}$ centromere A, noted
% $i_{n}^A$ is given by $i_{n}^A = 2n(M_k+1) + 2$. Similarly, we have:
% $$ \begin{aligned}
% i_{n}^B &= 2n(M_k+1) + M_k + 3 \\
% i_{nm}^A &= 2n(M_k+1) + 3 + m \\
% i_{nm}^B &= 2n(M_k+1) + M_k + 4 + m \\
% \end{aligned}$$




\end{document}

<!-- Local IspellDict: english -->

%%% Local Variables: 
%%% mode: latex
%%% TeX-master: t
%%% End: 
