\documentclass[a4paper,12pt]{article}


\usepackage[english]{babel}
\usepackage{ucs}
\usepackage[utf8x]{inputenc}
\usepackage{graphicx}
\usepackage{amsmath}
\usepackage{amsfonts}
\usepackage{amssymb}
\usepackage{upgreek}
\usepackage[colorlinks=True,%
linkcolor=black,%
citecolor=black,%
urlcolor=black]{hyperref}

\usepackage{natbib}
\usepackage{times}

\newcommand{\UM}{\upmu\mbox{m}}
\newcommand{\UMS}{\upmu\mbox{m}.\mbox{s}^{-1}}
\setlength{\voffset}{-1in}
\setlength{\hoffset}{-0.9in}
\setlength{\textwidth}{18cm}
\setlength{\textheight}{26cm}


\newcommand{\Pp}{prophase}
\newcommand{\PMp}{prometaphase}
\newcommand{\Mp}{metaphase}
\newcommand{\Ap}{anaphase}

\usepackage{natbib}






\title{Chromosome segregation model - detailed description}


\author{Guillaume Gay, Xavier Bressaud}
\date{}


\begin{document}
\bibliographystyle{plainnat}

\maketitle{}


\section*{Introduction}

This is a more detailed version of the kinetochore segregation model
published in \cite{gayJCB12}, which should be referred to for
all the experimental, biological and non-technical aspects of this
work.

 
\section{Definitions}
\label{sec:defs}

\subsection{State vector}

The mitotic spindle is described by the speeds and position along the
$x$ axis of two spindle pole bodies, $N$ chromosomes with two
centromeres and $M_k$ attachment sites per centromere.

Positions are noted as follow:
\begin{itemize}
\item The left and right spindle pole bodies ( SPBs ), $x_s^L$ and $x_s^R$
\item The $N$ centromeres, $x_n^A, \, x_n^B, n \in \{1,\cdots, N\}$
\item The $M_k$ attachment sites of each centromere, $x_{nm}^A, \,
  x_{nm}^B, n \in \{1, \cdots, N \}, m \in \{1, \cdots, M_k\}$
\end{itemize}
The speeds are noted with a dot: $dx / dt = \dot{x}$.

As all the interactions are assumed to be parallel to the spindle
axis, only the positions along this axis are considered, in a coordinate
system with its origin at the center of the spindle, which means that
$x_s^L(t) = - x_s^R(t)\, \forall t$.

The following interactions are considered:

\subsubsection{Forces at the right SPB :}
\begin{itemize}
\item Friction forces (viscous drag):  $F_s^f = -\mu_s \dot{x_s}^R$
\item Midzone force generators (applied at the right SPB): 
  $$F_{mid} = F_z\left(1 - (\dot{x}^R_s - \dot{x}_s^L)/V_z\right) =
  F_z\left(1 - 2\dot{x}^R_s / V_z\right) $$
\item Total kinetochore microtubules force generators:
  \begin{align}
    F_{kMT}^T = \sum_{n = 1}^{N}\sum_{m = 1}^{M_k} & - \rho_{nm}^A\,F_k\left( 1 -
      (\dot{x}^A_{nm} - \dot{x}^R_s)/V_k\right)\\
    & + \lambda_{nm}^A\,F_k\left(1 -
      (\dot{x}^A_{nm} + \dot{x}^R_s)/V_k\right)\\
    & - \rho_{nm}^B\,F_k\left( 1 -
      (\dot{x}^B_{nm} - \dot{x}^R_s)/V_k\right)\\
    & + \lambda_{nm}^A\,F_k\left(1 -
      (\dot{x}^B_{nm} + \dot{x}^R_s)/V_k\right)
  \end{align}
\end{itemize}

\subsubsection{Forces at centromere $An$}

\begin{itemize}
\item Drag: $F_c^f = -\mu_c \dot{x_n}^A$
\item Cohesin bond (Hook spring) restoring force exerted by centromere
  B: $$F_{BA} = -\kappa_c (x_n^A - x_n^B - d_0),
  \mbox{with } F_{BA} = - F_{AB}$$
\item Total visco-elastic bond between the centromere A and the attachment
  sites:
  $$ F_v^T = \sum_{m = 1}^{M_k} -\kappa_k(x_n^A - x_{nm}^A) 
  - \mu_k(\dot{x}_n^A - \dot{x}_{nm}^A) $$
\end{itemize}

\subsubsection{Forces at attachment site $Anm$}

\begin{itemize}
\item Visco-elastic bond between the centromere A and the
  attachment sites:
  $$F_v =  \kappa_k(x_n^A - x_{nm}^A) 
  + \mu_k(\dot{x}_n^A - \dot{x}_{nm}^A) $$
\item Kinetochore microtubules force generators:
  $$F_{kMT}^A = \rho_{nm}^A\,F_k\left(1 - (\dot{x}^A_{nm} -
    \dot{x}^R_s)/V_k\right) - \lambda_{nm}^A\,F_k\left(1 -
    (\dot{x}^A_{nm} - \dot{x}^L_s)/V_k\right) $$
\end{itemize}

Here, $\rho_{nm}^A$ and $\lambda_{nm}^A$ are two random variables that govern
the attachment state of the site $x_{nm}^A$, such that:
\begin{align}
  \label{eq:rholambda}
  \rho_{nm}^A &= 
  \begin{cases}
    1 &\text{if the site is attached to the right SPB}\\
    0 &\text{otherwise}\\
  \end{cases}\\
  \lambda_{nm}^A &=
  \begin{cases}
    1 &\text{if the site is attached to the left SPB}\\
    0 &\text{otherwise}\\
  \end{cases}
\end{align}

Note that $\rho_{nm}^A$ and $\lambda_{nm}^A$ are not independent, as
an attachment site can't be attached to both poles. To take this into
account, we can define the variable $p_{nm}^A = \rho_{nm}^A -
\lambda_{nm}^A$ such that:
\begin{equation}
  \label{eq:rholambda}
  p_{nm}^A = 
  \begin{cases}
    - 1 &\text{if the site is attached to the left SPB}\\
    0 &\text{if the site is not attached}\\
    1 &\text{if the site is attached to the right SPB}\\
  \end{cases}\\
\end{equation}
We have:
\begin{align}
  \lambda_{nm}^A &= p_{nm}^A\left(p_{nm}^A - 1\right)/2\\
  \rho_{nm}^A &= p_{nm}^A\left(p_{nm}^A + 1\right)/2
\end{align}
We also define $N_n^{AL}$ and $N_n^{AR}$ as the number of ktMTs of
centromere A attached to the left and right SPBs, respectively:
\begin{equation}
  \label{eq:NAL}
  N_n^{AL} = \sum_{m = 1}^{M_k}\lambda_{nm}^A \mbox{ and }%
  N_n^{AR} = \sum_{m = 1}^{M_k}\rho_{nm}^A 
\end{equation}
Note that $N_n^{AL} + N_n^{AR} \leq M_k\, \forall\, p_{nm} $
The same definitions apply for the centromere B and left SPB.

\subsection{Set of first order coupled equations}

In the viscous nucleoplasm, inertia is negligible. Newton first
principle thus reduces to: $ \sum F = 0 $, all the equations are
gathered together in the system of equations:
$$
\mathbf{A}\dot{X} + \mathbf{B}X + C = 0
$$
The vector $X$ has $1 + 2N(M_k + 1)$ elements and is defined as follow:
\begin{equation*}
  X = \{x_s^R, \{x_n^A, \{x_{nm}^A\},  x_n^B,% 
  \{x_{nm}^B \}\}\}\mbox{ with } n \in 1 \cdots N %
  \mbox{ and } m \in 1 \cdots M_k
\end{equation*}
With this order, the index of the $n^{th}$ centromere A, noted
$i_{n}^A$ is given by $i_{n}^A = 2n(M_k+1) + 2$. Similarly, we have:
$$ \begin{aligned}
i_{n}^B &= 2n(M_k+1) + M_k + 3 \\
i_{nm}^A &= 2n(M_k+1) + 3 + m \\
i_{nm}^B &= 2n(M_k+1) + M_k + 4 + m \\
\end{aligned}$$

To simplify the equations, we set $F_k$ as unit force and $V_k$ as unit
speed, thus $F_k/V_k = 1$. From the above we have:
\begin{equation}
  \begin{aligned}
    A = &% 
    \begin{pmatrix}
      a_{1,1} & \hdotsfor{1} & a_{1, i_{nm}^A} &%
      \hdotsfor{1} & a_{1, i_{nm}^B}\\
      %%%%% Centromere A %%%%
      \hdotsfor{1} & a_{i_{n}^A, i_{n}^A}& %
      a_{i_{n}^A, i_{nm}^A} & \hdotsfor{1}\\
      %%%%% Att. Site A %%%%
      a_{i_{nm}^A, 1} & a_{i_{nm}^A, i_{n}^A} & a_{i_{nm}^A, i_{nm}^A}&%
      \hdotsfor{2}\\
      %%%%% Centromere B %%%%
      \hdotsfor{3} & a_{i_{n}^B, i_{n}^B}& a_{i_{n}^B, i_{nm}^B}\\
      %%%%% Att. Site B %%%%
      a_{i_{nm}^B,1} & \hdotsfor{2} &  a_{i_{n}^B, i_{nm}^B} &%
      a_{i_{nm}^B, i_{nm}^B}\\
    \end{pmatrix}\\
    =  & \begin{pmatrix}
      %%%%% SPB %%%%
      -\mu_s - F_z/V_z - \sum (p_{nm}^A + p_{nm}^B)& \hdotsfor{1} & p_{nm}^A &%
      \hdotsfor{1} &  p_{nm}^B\\
      %%%%% Centromere A %%%%
      \hdotsfor{1} &  -\mu_c - M_k \mu_k& \mu_k & \hdotsfor{2}\\
      %%%%% Att. Site A %%%%
      p_{nm}^A & \mu_k & - \mu_k + p_{nm}^A & \hdotsfor{2}\\
      %%%%% Centromere B %%%%
      \hdotsfor{3} & -\mu_c - M_k \mu_k & \mu_k\\
      %%%%% Att. Site B %%%%
      p_{nm}^B & \hdotsfor{2} & \mu_k & - \mu_k + p_{nm}^B \\
    \end{pmatrix}, \\
    B = &%
    \begin{pmatrix}
      \,0\, & \hdotsfor{4}\\
      \hdotsfor{1} & - \kappa_c - M_k \kappa_k & \kappa_k &%
      \kappa_c & \hdotsfor{1} \\
      \hdotsfor{1} & \kappa_k & -\kappa_k &  \hdotsfor{2}\\
      \hdotsfor{1} & \kappa_c & \hdotsfor{1} &%
      -\kappa_c - M_k \kappa_k & \kappa_k \\
      \hdotsfor{3}  & \kappa_k & - \kappa_k\\
    \end{pmatrix}% 
    \quad \mbox{and} \quad C = %
    \begin{pmatrix}
      0\\
      0\\
      d_0\\
      0\\
      d_0\\
      0\\
    \end{pmatrix}
  \end{aligned}
\end{equation}

\section{Continuous time Markov chain description\\
 of the attachment -- detachment process}
\label{sec:markov}


\subsection{Attachment and detachment rates}

The attachment sites attach or detach stochastically with rates $k_a^{R/L}$
and $k_d$, i.e:
\begin{equation}
  \begin{aligned}
    p_{nm}^A = 1 &\xrightarrow{\quad k_d \quad}& p_{nm}^A = 0%
    &\xrightarrow{\quad k_a^R \quad}& p_{nm}^A = 1\\
    p_{nm}^A= -1 &\xrightarrow{\quad k_d \quad}& p_{nm}^A = 0%
    &\xrightarrow{\quad k_a^L \quad}& p_{nm}^A = -1\\
  \end{aligned}
\end{equation}
The detachment rate depends on the position of the attached site with
respect to the chromosome center:
\begin{equation}
  \label{eq:k_d}
  k_d = k_ad_\alpha / d, \mbox{with } d = \left| x^A_{nm} - %
  \left(x^A_{n}+ x^B_{n}\right) / 2 \right|
\end{equation}
The attachment rate depends on the state of the other attachment
sites:
\begin{equation}
  \label{eq:k_a}
  k_a^R = k_a\left( 1/2 + \beta\frac{N_n^{AR} - N_n^{AL}}%
    {2\left(N_n^{AR} + N_n^{AL}\right)}\right)
\end{equation}
  
\subsection{Discrete state-space approximation of the stochastic
  process}
\label{sec:Defintions}

The coupling of $k_d$ with the mechanical aspects of the global model
prevents a straightforward description of the model as a continuous
time Markov chain. This section presents a discrete approximation, as
a first attempt.

We now consider only one chromosome with two centromeres and $M_k$
attachment sites. The state of this model is then completely specified
by the four random variables $N^{AL}, N^{AR}, N^{BL}, N^{BR}$ as
defined in equation \ref{eq:NAL}. 

Considering for example the variable $N^{AL}$, we have: $N^{AL} $  
\begin{equation}
    N^{AL} \xrightarrow{\quad N^{AL}k_d \quad} N^{AL}- 1%
\end{equation}

We define the variables $P^A$ (and similarly $P^B$):
\begin{equation}
  \label{eq:PA}
  P^A = N^{AL}, N^{AR} = \sum_{M_k} p_m^A
\end{equation}
With $p_m^A$ as defined in equation \ref{eq:rholambda}. $P^A$ can be
viewed as the force balance at the centromere A. 


In the model, the detachment rate $k_d$ depends on the distance
between the centromeres. At equilibrium, this distance is proportional
to the net force applied to the chromosome by the attached
microtubules, i.e. to   $P^A  + P^B$  
\begin{equation}
  \label{eq:def_kf_markov}
  k_d =%
  \begin{cases}
    k_a d_{\alpha} / d_{eq}  \mbox{ with } d_{eq} = 1 + \frac{P^A  +%
      P^B}{\kappa_c} & \mbox{ if } P^AP^B < 0 \\
    k_a * d_{\alpha} &  \mbox{ if }  P^AP^B \geq 0\\
  \end{cases}
\end{equation}

        % #With Fk as unit force and d_0 the unit distance:
        % if sign( forceA * forceB ) <= 0:
        %     d_eq =  1. + abs(forceA - forceB) / kappa 
        %     k_d = k_a * d_alpha / d_eq
        % else:
        %     k_d = k_a * 5. 




% On veut evaluer les taux de transition en fonction de l'etat.
% Pour chacune des composantes, le raisonnement est le même:

 
% Regardons NAG

 
% NAG peut diminuer de 1. 

 
% Cela arrive avec un taux proportionnel a NAG-1.
% C'est aussi fonction de la "distance". 
% On ecrit: Kd=Ka d_alpha/ d
% ou d_alpha de l'ordre de 3 d_0 (d_0 a lequilibre)
% et d la distance réelle entre les deux kinteomachin. 
% On ecrit d=(1 + 3M/8) d_0, 
% de maniere a ce que si M=0 on obtienne d_0 et si M=8, 4 d_0
% Pour evaluer M on distingue les cas ou les forces exercées sur A et B
% sont de même signe, de signes opposés 
% Si elles sont de signe opposé, on pose M= somme des val abs de ces forces
% cad: M= abs(NAG + NBD - NAD - NBG)
% Sinon on pose M=-4/3 pour avoir d= d_0/2  
% Si les deux sont nulles, on s'en fout puisque personne attaché. 

 

 
% NAD+NAG peut augmenter de 1.

 
% Le taux est proportionnel au nombre de recepteurs libres, i.e. a N-NAG-NAD
% La constante est Ka
% Mais alors c'est NAG ou NAD qui augmente de 1 ; la proba que ce soit NAG
% est donnée par 1/2 + beta x (NAG-NBG) / 2 (NAG-1 + NAD-1)



\end{document}

We give in the following
the explicit expressions of the equations.

\subsection{Centromere}

\begin{equation}\label{eq:fkt}
  \begin{aligned}
    \mu_i \dot{x}_n &= -\kappa_i(x_n - x_n^{\dagger}-d_0) +
    \sum_m\left[ -\kappa_o(x_n - x_{nm}) - \mu_o(\dot{x}_n -
      \dot{x}_{nm})\right]\\
    \mu_i \dot{x}_n^{\dagger} &= -\kappa_i(x_n^{\dagger} - x_n + d_0)
    + \sum_m\left[ -\kappa_o(x_n^{\dagger} - x_{nm}^{\dagger}) -
      \mu_o(\dot{x}_n^{\dagger} - \dot{x}_{nm}^{\dagger})\right]\\
  \end{aligned}
\end{equation}

\subsection{Attachment sites}

\begin{equation}\label{eq:fMk}
  \begin{aligned}
    \mu_{o}(\dot{x}_{nm} - \dot{x}_{n}) &= -\kappa_o(x_{nm} -x_n) +
    \alpha_{nm}F_k\left(1 - \frac{\dot{x}_{nm} -
        \dot{x}_s}{V_k}\right) - \beta_{nm}F_k\left(1 -
      \frac{\dot{x}_s^{\dagger} - \dot{x}_{nm}}{V_k}\right)\\
    \mu_{o}(\dot{x}_{nm}^{\dagger} - \dot{x}^{\dagger}_{n}) &=
    -\kappa_o(x_{nm}^{\dagger} -x_n^{\dagger}) - \alpha_{nm}^{\dagger}F_k\left(1
      - \frac{\dot{x}_{s}^{\dagger} - \dot{x}_{nm}^{\dagger}}{V_k}\right) +
    \beta_{nm}^{\dagger}F_k\left(1 -
      \frac{\dot{x}_{nm}^{\dagger} - \dot{x}_s}{V_k}\right)\\
  \end{aligned}
\end{equation}

With $\alpha_{nm} = 1$ if the attachment site is attached to the
correct SPB, 0 otherwise and   $\beta_{nm} = 1$ if it is attached
erroneously\footnote{This means that the attachment of this site
  contributes to an incorrect attachment of the chromosome, either
  merotelic, syntelic or monotelic}, 0 otherwise.

\subsection{Spindle Pole Bodies}

\begin{equation}\label{eq:spb}
  \begin{aligned}
    \mu_s\dot{x}_s &= F_z\left(1 - \frac{\dot{x}_s -
        \dot{x}_s^{\dagger}}{V_z}\right) + \sum_n\sum_m\left[-
      \alpha_{nm}F_k\left(1 - \frac{\dot{x}_{nm} -
          \dot{x}_s}{V_k}\right) - \beta_{nm}^{\dagger}F_k\left(1 -
        \frac{\dot{x}_{nm}^{\dagger} - \dot{x}_s}{V_k}\right)\right]\\
    \mu_s\dot{x}_s^{\dagger} &= - F_z\left(1 - \frac{\dot{x}_s -
        \dot{x}_s^{\dagger}}{V_z}\right) +
    \sum_n\sum_m\left[\alpha_{nm}^{\dagger}F_k\left(1 -
        \frac{\dot{x}_{s}^{\dagger} -
          \dot{x}_{nm}^{\dagger}}{V_k}\right) + \beta_{nm}F_k\left(1 -
        \frac{\dot{x}_s^{\dagger} - \dot{x}_{nm}}{V_k}\right)\right]\\
  \end{aligned}
\end{equation}
$x_s^{\dagger} = -x_s\, \forall\, t \Rightarrow$
%\begin{equation}
\begin{multline}
  2\mu_s \dot{x}_s = 2F_z\left(1 - \frac{2\dot{x}_s}{V_z}\right) +
  \sum_n\sum_m\Biggl[- \alpha_{nm}F_k\left(1 - \frac{\dot{x}_{nm} -
      \dot{x}_s}{V_k}\right) - \beta_{nm}^{\dagger}F_k\left(1 -
    \frac{\dot{x}_{nm}^{\dagger} -
      \dot{x}_s}{V_k}\right)\\
  - \alpha_{nm}^{\dagger}F_k\left(1 -
    \frac{-\dot{x}_{s} -
      \dot{x}_{nm}^{\dagger}}{V_k}\right) - \beta_{nm}F_k\left(1 -
    \frac{- \dot{x}_s - \dot{x}_{nm}}{V_k}\right)\Biggr]
\end{multline}
% \end{equation}



<!-- Local IspellDict: english -->

%%% Local Variables: 
%%% mode: latex
%%% TeX-master: t
%%% End: 
